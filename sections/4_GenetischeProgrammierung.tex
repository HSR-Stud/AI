\section{Genetische Algorithmen}
Genetische Algorithmen gründen auf der Formalisierung der wichtigsten
Prozesse und Begriffe aus der Evolution:
\begin{itemize}
	\item Kreuzung
	\item Mutation
	\item Fitness
\end{itemize}
Keine Muster für Zuordnung, dafür Funktionen und eine Fitnesslandschaft.
Die Fitnesslandschaft ersetzt die Korrelationsmatrix des
Hopfield-Netzwerks.
\subsection{Ablauf}
\begin{enumerate}
	\item Zufällige Chromosomen erstellen
	\item Gemäss Fitness Wahrscheinlichkeit für nächste Kreuzung bestimmen
	\item Zwei Chromosomen mit skalierter Wahrscheinlichkeit auswählen
		und an zufälliger Stelle teilen und je mit dem anderen
		Gegenstück zusammensetzen.
	\item Die neu erstellten Chromosomen dürfen leicht mutiert werden.
\end{enumerate}
\subsection{Genetische Programmierung}
Anwendung von genetischen Algorithmen auf Programme selbst, bei denen
der Zweck definiert ist. Die Fitnessfunktion entspricht dabei dem
Erfüllen der Aufgabe.

